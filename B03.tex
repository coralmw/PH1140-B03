\documentclass[10pt]{iopart}
\usepackage{graphicx}
%Uncomment next line if AMS fonts required
%\usepackage{iopams}
\begin{document}

\title[PH1140 Report - T GM Parks]{PH1140 Scientific Skills: Report Poisson statistics using $\gamma$ radiation}

\author{T GM Parks\\28th December 2014}

\begin{abstract}
The distribution of the number of $\gamma$ emissions observed from a radioactive source in a constant time was shown to be modeled by a Poisson distribution. We used a GM tube to count $\gamma$ emissions from a Caesium-137 source for 10 seconds 100 times, and found that the mean, standard deviation and the 'shape' of the distribution of counts was modelled well by a Poisson distribution.
\end{abstract}

%\maketitle

\section{Introduction}

\section{Experimental procedure}

A radioactive source was placed in a Lead cage, ~10cm below the apeture for a Giger-Muller tube. No sheilding was placed bwteen the source and the detector, and the experimental appratus was not changed. The GM tube was connected to a timer counter set for a 40kV potential and a 10 second count.

A 10 second count of radioactive emissions was taken 100 times and recorded.

\section{Analysis of the data}

\subsection{Comparison with Poisson distribution}

The data was analyzed using python+numpy. The mean, standard deviation, and standard error of the mean were calculated
mean:49.161904761904765
standard deviation: 7.549990614341448
error of mean: 0.5222437709993091


\end{document}
