%This is the source file for a LaTeX version of the example report
%for the Communications Skills section of PH1140. It uses the
%iopart.cls class file from the Institute of Physics. This class
%file and its options can be obtained from http://authors.iop.org
%by following the link to 'LaTeX Guidelines and Class File' and
%should be downloaded into the same directory as your LaTeX source
%code.
%
% S J Flockton 8th November 2007

\documentclass[10pt]{iopart}
\usepackage{graphicx}
%Uncomment next line if AMS fonts required
%\usepackage{iopams}
\begin{document}

\title[PH1140 Report - A N Other]{PH1140 Scientific Skills: Report
\\Thermal radiation - the inverse square law}

\author{A N Other\\27th November 2003}

\begin{abstract}
The inverse square law for the radiation intensity from a
Stefan-Boltzmann lamp has been demonstrated to hold for distances
greater than 15cm using a Pasco thermal radiation sensor. At
distances below 15cm the effect of the finite size of the lamp
element has been clearly observed and the effective position of
the filament shown to be $(0.40 \pm  0.08)$cm behind its
geometrical position.
\end{abstract}

%\maketitle

\section{Introduction}

The inverse square law describing the decrease in intensity of
thermal radiation with distance from a point source follows
immediately from conservation of energy \cite{yandf,nave}. The law
can be readily tested using a suitable lamp and a radiation
sensor. In the experiment described here commercial equipment in
the form of a lamp and radiation detector produced by Pasco
\cite{pasco} was used.

The lamp is described as a Stefan-Boltzmann lamp. Its temperature
can be determined accurately from the voltage and current supplied
to the lamp in order to investigate the Stefan-Boltzmann law
although this feature was not used in the experiment described
here.

The radiation sensor measures intensities in the wavelength range
$0.64$ to $30\mu$m and provides relative intensities of the
thermal radiation emitted by objects to which it is pointed.

\section{Experimental procedure}

The equipment was set up as shown in Figure \ref{setup}. The meter
rule was taped to the table and the lamp fixed at one end of the
rule with the centre of the filament set on zero. The radiation
sensor was set at the same height as the filament and the two were
aligned to be coaxial.

\begin{figure}
  \centering
  \includegraphics[width=\textwidth]{setup}\\
  \caption{Experimental arrangement showing the radiation sensor
(left) and the lamp (right). (Figure 2.1 from reference
\cite{pasco}).}\label{setup}
\end{figure}

The Manual gave the sensor position as about 1mm from the end of
the detector housing. Measurements were made of the position of
the front of the base along the ruler allowing the sensor position
to be determined by subtracting $22$mm from the base position as
shown in Figure \ref{distdet}. The position uncertainty was
estimated to be $\pm1$mm. The filament size is about $3$mm.

\begin{figure}
  \centering
  \includegraphics[width=0.9\textwidth]{distdet}\\
  \caption{Determination of the distance between the lamp and the
sensor.}\label{distdet}
\end{figure}

With the lamp off, the millivoltmeter background reading was $(0.0
\pm 0.1)$mV, independent of position, although care was needed
since a nearby hand affected the reading. With the meter
disconnected the reading was $-0.1$mV. A number of readings were
made at different positions with the lamp off, all producing a
value of zero or $-0.1$mV, consistent with the above meter offset.
The mean of these readings, $-0.06$mV, was therefore defined as
$R_0$.

Trial readings showed that with a voltage of $10$V applied to the
lamp the sensor produced an output of about $150$mV with a
separation of lamp and sensor of $2.5$cm, $2$mV at $25$cm and
$0.1$mV (consistent with background) at $90$cm.

It became clear during these trial readings that a heat shield was
needed between readings to avoid heating of the sensor holder.
Such a shield, consisting of two pieces of silvered cardboard, was
used. The final set of reading was taken quickly to reduce this
heating effect and is recorded in Table \ref{table1}.

\begin{table}
  \centering
  \caption{Radiation level as a function of distance between source
and sensor.\\}\label{table1}
\begin{tabular}{|c|c||c|c|}
\hline
 % % after \\: \hline or \cline{col1-col2} \cline{col3-col4} ...
Separation  & Radiation sensor output  & Separation
 & Radiation sensor output \\
X (cm)&R (mV)&X(cm)&R (mV)\\
\hline $2.5 \pm 0.1$ & $ 141.9 \pm 0.1$ & $ 18.0 \pm 0.1$ & $ 3.8 \pm 0.1$\\
$3.0$ & $91.8$ & $20.0$ & $3.1$\\
$3.5$ & $80.3$ & $25.0$ & $2.0$\\
$4.0$ & $59.4$ & $30.0$ & $1.3$\\
$4.5$ & $48.6$ & $35.0$ & $1.0$\\
$5.0$ & $40.4$ & $40.0$ & $0.7$\\
$6.0$ & $29.6$ & $45.0$ & $0.6$\\
$7.0$ & $21.7$ & $50.0$ & $0.5$\\
$8.0$ & $17.0$ & $60.0$ & $0.3$\\
$9.0$ & $13.8$ & $70.0$ & $0.2$\\
$10.0$ & $11.5$ & $80.0$ & $0.2$\\
$12.0$ & $8.2$ & $90.0$ & $0.1$\\
$14.0$ & $6.1$ & $100.0$ & $0.1$\\
$16.0$ & $4.7$&&\\
\hline
\end{tabular}
\end{table}

\section{Analysis of the data}

\subsection{Comparison with inverse square law}

\begin{figure}
  \centering
  \includegraphics[width=0.8\textwidth]{raw}\\
  \caption{Intensity as a function of X.}\label{raw}
\end{figure}

The data were analysed using gnuplot. A simple plot of $R-R_0$
versus $X$ is shown in Figure \ref{raw} and of $R-R_0$ versus
$1/X^2$ in Figure \ref{1overxsq} for all data points. The
penultimate point in Figure \ref{1overxsq} appears to be badly
measured but even ignoring this point the data are not well-fitted
by a straight line. Thus the inverse square law does not appear to
be describing the behaviour well.

\begin{figure}
  \centering
  \includegraphics[width=0.8\textwidth]{1overxsq}\\
  \caption{Intensity as a function of $1/X^2$.}\label{1overxsq}
\end{figure}

A possible reason for this observation is that the source is not
point-like (as noted earlier it is about $3$mm in size). If this
is the cause of the deviation from inverse square law behaviour,
data points at large $X$ might still be expected to be a good fit
to $1/X^2$. Indeed when only those points with $X>15$cm are
plotted (Figure \ref{select_1overxsq}) the fit is much better. The
straight line passes through all the error bars, implying that
they are in fact longer than one standard deviation, and have been
overestimated.

\begin{figure}
  \centering
  \includegraphics[width=0.8\textwidth]{select_1overxsq}\\
  \caption{Intensity as a function of $1/X^2$  for large X.}\label{select_1overxsq}
\end{figure}

\subsection{Compensation for finite size of source}

Another possible reason for the curve in the points shown in
Figure \ref{1overxsq} is that because of its finite size, the
effective source was not in the geometrical centre of the lamp,
leading to an offset in the distance measurements. This was
investigated by comparing the fit obtained to  $R=A/X^2$ with that
obtained to $R=A/(X+c)^2$. These are compared in Figures
\ref{fit_raw} and \ref{fit_raw_offset}.

\begin{figure}
  \centering
  \includegraphics[width=0.8\textwidth]{fit_raw}\\
  \caption{Fit of intensity to function  $A/X^2$.}\label{fit_raw}
\end{figure}

\begin{figure}
  \centering
  \includegraphics[width=0.8\textwidth]{fit_raw_offset}\\
  \caption{Fit of intensity to function  $A/(X+c)^2$.}\label{fit_raw_offset}
\end{figure}


Fitting to the first equation gave a result of  $A = (90 \pm
1)\times 10^1$~mV~cm$^2$, while the second, which can be seen to
fit the data much better, gave results of $A  = (116 \pm  6)\times
10^1$~mV~cm$^2$ and $c = (0.40 \pm 0.08)$cm. It seems, therefore,
that the effective centre of the lamp is about $0.4$cm behind
where it was originally thought to have been.

\section{Summary}

The radiation intensity from a lamp was measured as a function of
distance, $X$, from a lamp. For  $X$ larger than about $15$cm a
good fit was obtained to the inverse square law. For smaller
values of $X$ deviations were observed, due either to a systematic
error in the measurement of the distance or to the finite size of
the lamp element. A simple model with $R-R_0$  proportional to
$1/(X+c)^2$ gave an excellent fit for all the data with  $c =
(0.40 \pm 0.08)$cm. From this one cannot conclude whether the
deviation was due to the finite size of the lamp or a systematic
error in the distance measurement; both are consistent with the
observed data. A small long term drift of the output of the lamp
was observed, but this was insignificant compared with the other
sources of uncertainty.

\section*{References}

\begin{thebibliography}{10}
\bibitem{yandf} Young H D and Freedman R A, {\it University Physics:
with Modern Physics} 11th edition. New York: Addison Wesley, 2004,
p567.
\bibitem{nave} Nave R, ``Inverse Square Law'',
http://hyperphysics.phy-astr.gsu.edu/hbase/forces/isq.html,
retrieved 24 November 2003.
\bibitem{pasco} Instruction Manual and Experiment Guide for the PASCO
scientific Model TD-8553/8554A/8555. PASCO scientific, 10101
Foothills Blvd., Roseville, CA, USA, 1988.
\end{thebibliography}

\end{document}
